\section{RIOT}
\label{sec:riot}

Το \textit{RIOT} \cite{bib:riot} είναι ένα λειτουργικό σύστημα βασισμένο σε μικροπυρήνα ανοιχτού κώδικα, σχεδιασμένο για να ανταποκρίνεται στις απαιτήσεις των συσκευών IoT και άλλων ενσωματωμένων συσκευών. Αυτές οι απαιτήσεις περιλαμβάνουν ένα πολύ χαμηλό αποτύπωμα μνήμης (της τάξης των λίγων kilobytes), υψηλή ενεργειακή απόδοση, δυνατότητες σε πραγματικό χρόνο, υποστήριξη για ένα ευρύ φάσμα υλικού χαμηλής κατανάλωσης ενέργειας, στοίβες επικοινωνίας για ασύρματα και ενσύρματα δίκτυα.

Το RIOT παρέχει έναν μικροπυρήνα, πολλαπλές στοίβες δικτύου και βοηθητικά προγράμματα που περιλαμβάνουν κρυπτογραφικές βιβλιοθήκες, δομές δεδομένων, ένα τερματικό και άλλα. Το RIOT υποστηρίζει ένα ευρύ φάσμα αρχιτεκτονικών μικροελεγκτών, αισθητήρων και διαμορφώσεων για ολόκληρες πλατφόρμες, π.χ. Atmel SAM R21 Xplained Pro, Zolertia Z1, STM32 Discovery Boards κ.λ.π. σε όλο το υποστηριζόμενο υλικό (πλατφόρμες 32-bit, 16-bit και 8-bit). Το RIOT παρέχει τη δυνατότητα για προγραμματισμό εφαρμογών σε ANSI C και C++, με multreadreading, χρονοδιακόπτες, κ.α.

Στην παρούσα εργασία χρησιμοποιήθηκε καθώς όλη η μοντελοποίηση έγινε με στόχο την παραγωγή κώδικα ο οποίος θα εκτελείται σε RIOT.

\begin{figure}[!ht]
  \centering
  \includegraphics[width=0.4\textwidth]{./images/chapter4/riot.png}
  \caption[Λειτουργικό σύστημα RIOT.]{Λειτουργικό σύστημα RIOT.\footnotemark}
  \label{fig:riot}
\end{figure}

\footnotetext{\url{https://www.riot-os.org/}}