\chapter{Επισκόπηση της Ερευνητικής Περιοχής}
\label{chapter:sota}

Στο κεφάλαιο αυτό γίνεται μία σύντομη αναφορά σε ήδη υπάρχουσες μοντελο-κεντρικές υλοποιήσεις στον τομέα του IoT. Οι υλοποιήσεις αυτές παρέχουν είτε γλώσσες μοντελοποίησης γενικού σκοπού, είτε γλώσσες συγκεκριμένου τομέα. Αν και υπάρχουν αρκετές σχετικές δημοσιεύσεις τα τελευταία 7 περίπου χρόνια, η πρώτη απόπειρα μοντελοποίησης του RIOT έγινε στα μέσα του 2020 \cite{bib:karaduman_2020}.

Οι \citet{bib:salihbegovic_2015} δημιούργησαν μια Οπτική γλώσσα μοντελοποίησης συγκεκριμένου τομέα (VDSML), βασισμένη σε JavaScript\footnote{JavaScript, \url{https://www.javascript.com/}}, η οποία παρέχει μια διεπαφή χρήστη για την σχεδίαση ενός IoT συστήματος. Η εκτέλεση των παραχθέντων αρχείων γίνεται με τη χρήση του OpenHAB\footnote{OpenHAB, \url{https://www.openhab.org/}}.

Η πρώτη ολοκληρωμένη υλοποίηση έγινε από τους \citet{bib:harrand_2016} δημιουργώντας την ThingML, μια γλώσσα συγκεκριμένου τομέα (DSL) για τη μοντελοποίηση IoT συστημάτων χρησιμοποιώντας πεπερασμένα αυτόματα (state machines). Η ThingML φαίνεται πως είναι ένα καλό και χρήσιμο εργαλείο, το οποίο χρησιμοποιείται και συντηρείται μέχρι και σήμερα.

Οι \citet{bib:berrouyne2019} δημιούργησαν ένα εργαλείο, το CyprIoT, για τη μοντελοποίηση και τον έλεγχο διαδικτυακών IoT εφαρμογών. Παρέχουν λοιπόν δύο γλώσσες. Η πρώτη στοχεύει στον σχεδιασμό ενός δικτύου με έναν πιο ευανάγνωστο και υψηλότερου επιπέδου τρόπο, χρησιμοποιώντας μάλιστα την ThingML για την μοντελοποίηση των διασυνδέσεων. Η δεύτερη, είναι μια γλώσσα πολιτικής (Policy Language), στοχεύει δηλαδή στη δημιουργία κανόνων για τον έλεγχο των διαδικτυακών συνδέσεων των συσκευών. Επίσης, παρέχουν και ένα εργαλείο παραγωγής κώδικα.

Σε αυτό το σημείο θα αναφερθούν και τρεις ακόμα υλοποιήσεις, μέσω των οποίων γίνεται μοντελοποίηση συγκεκριμένης πλατφόρμας (\EN{Platform-specific modeling}), και άρα είναι άρρηκτα συνδεδεμένες με το αντικείμενο της παρούσας διπλωματικής. Οι υλοποιήσεις αυτές αποσκοπούν στην μετέπειτα ύπαρξη ενός πιο γενικού, ανεξάρτητου της πλατφόρμας (Platform independent) εργαλείου, για τη μοντελοποίηση της επικοινωνίας ασύρματου δικτύου αισθητήρων (WSN) σε ένα IoT σύστημα. 

Αρχικά, οι \citet{bib:hussein_2018} δημιούργησαν μια γλώσσα μοντελοποίησης συγκεκριμένου τομέα, για το λειτουργικό TinyOS\footnote{TinyOS, \url{http://www.tinyos.net/}}. Μέσω αυτής, γίνεται η μοντελοποίηση των διασυνδέσεων των συσκευών ενός IoT συστήματος, και στη συνέχεια η παραγωγή κώδικα nesC \cite{bib:gay_2003} για την εκτέλεση συγκεκριμένων λειτουργιών στο λειτουργικό TinyOS. 

Επόμενη υλοποίηση ήταν η επέκταση μιας προηγούμενης \cite{bib:durmaz_2017} πιο περιορισμένης, η οποία παρείχε τη μοντελοποίηση εφαρμογών σε ContikiOS\footnote{ContikiOS, \url{https://www.contiki-ng.org/}}. Στη νέα υλοποίηση από τους \citet{bib:karaduman_2019}, επεκτείνεται το μετα-μοντέλο ώστε να υποστηρίζει επιπλέον εξαρτήματα που χρησιμοποιούνται σε ένα IoT σύστημα, δημιουργείται ένας γραφικός συντάκτης, και παρέχεται και ένα εργαλείο κανόνων για την παραγωγή κώδικα.

Η πιο πρόσφατη υλοποίηση \cite{bib:karaduman_2020}, ήταν μια μοντελοκεντρική προσέγγιση για την ανάπτυξη IoT συστημάτων για το λειτουργικό RIOT\footnote{RIOT OS \url{https://www.riot-os.org/}}, με το οποίο θα ασχοληθούμε και στην παρούσα εργασία. Δημιουργήθηκε ένα μετα-μοντέλο για το λειτουργικό RIOT, και με βάση αυτό παράγεται μία γλώσσα μοντελοποίησης συγκεκριμένου τομέα (DSML). Τέλος, μέσω συγκεκριμένων κανόνων, παράγεται κώδικας για την εκτέλεση συγκεκριμένων λειτουργιών στο λειτουργικό RIOT. Η υλοποίηση αυτή, μοντελοποιεί ένα μεγάλο κομμάτι του RIOT, και μάλιστα φαίνεται να έχει ποιοτικά αποτελέσματα, όσον αφορά τον παραγόμενο κώδικα. Ωστόσο, επικεντρώνεται κυρίως στον τρόπο επικοινωνίας μεταξύ των συσκευών στο δίκτυο. Στην παρούσα εργασία, πέρα από τη μοντελοποίηση των βασικών στοιχείων του RIOT, και την παραγωγή κώδικα, θα παρέχουμε στον χρήστη πληροφορία σχετικά με τον τρόπο διασύνδεσης των εξαρτημάτων (μικροελεγκτών, αισθητήρων, ενεργοποιητών).