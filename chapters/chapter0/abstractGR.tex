\begin{center}
  \centering

  \vspace{0.5cm}
  \centering
  \textbf{\Large{Περίληψη}}
  \phantomsection
  \addcontentsline{toc}{section}{Περίληψη}

  \vspace{1cm}

\end{center}

Το διαδίκτυο των πραγμάτων (Internet of Things ή IOT) είναι ένας κλάδος που εξελίσσεται ραγδαία ειδικά τα τελευταία χρόνια. Άρα υπάρχει η δυνατότητα ανάπτυξης όλο και περισσότερων εφαρμογών, χρήσιμες για πολλούς ανθρώπους, είτε έχουν να κάνουν με απλές λειτουργίες σε συστήματα αυτοματισμού, είτε με μεγαλύτερης κλίμακας εφαρμογές στη βιομηχανία. Επομένως, όλο και περισσότερος κόσμος επιθυμεί να ασχοληθεί με τον τομέα αυτό, και μια μεγάλη μερίδα του είναι άτομα ακατάρτιστα. Αυτό έχει ως αποτέλεσμα να χάνουν της δυνατότητες που το IoT μπορεί να προσφέρει.

Η μοντελοστρεφής μηχανική (Model Driven Engineering ή MDE), έρχεται να δώσει λύσει σε αυτό το πρόβλημα, καθώς μπορεί να παρέχει σε αυτά τα άτομα την ανάπτυξη IoT συστημάτων σε ένα πιο αφαιρετικό επίπεδο, το οποίο είναι πιο φιλικό προς τον απλό χρήστη.

Μέσω της παρούσας διπλωματικής εργασίας, δίνεται η δυνατότητα σε κάποιον χρήστη να μοντελοποιήσει το σύστημα που επιθυμεί να υλοποιήσει, μέσω ενός εργαλείου κειμένου περιγραφής συσκευών και των μεταξύ τους συνδέσεων. Ταυτόχρονα, παρέχει την αυτόματη παραγωγή κώδικα για μια πληθώρα IoT συσκευών, προσαρμοσμένη στα χαρακτηριστικά που επιθυμεί ο χρήστης να έχει το σύστημά του, και έτσι του δίνεται έτοιμη μια βασική υλοποίηση κάποιων λειτουργιών, χωρίς να χρειάζεται να γράψει καθόλου κώδικα. Μάλιστα, ο κώδικας που παράγεται είναι χαμηλού επιπέδου, καθώς έχει σχεδιαστεί σύμφωνα με τις απαιτήσεις ενός λειτουργικού συστήματος πραγματικού χρόνου (Real Time Operating System ή RTOS), το RIOT.