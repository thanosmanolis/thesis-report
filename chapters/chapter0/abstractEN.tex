{\fontfamily{cmr}\selectfont

\phantomsection
\addcontentsline{toc}{section}{Abstract}


\begin{center}
  \centering
  \textbf{\Large{Title}}
  \vspace{0.5cm}

  %\textbf{\large{Simultaneous Localization \& Mapping Combining \\Particle Filters, Critical Rays Scan Match \& Topological Information}}
  \textbf{\large{Model-driven development for low-consumption real-time IOT devices}}

  \vspace{1cm}

  \centering
  \textbf{Abstract}
\end{center}


\begin{flushright}
  \vspace{2cm}
  Athanasios Manolis
  \\
  Intelligent Systems and Software Engineering Labgroup (ISSEL)
  \\
  Electrical \& Computer Engineering Department,
  \\
  Aristotle University of Thessaloniki, Greece
  \\
  September 2021
\end{flushright}

}

Internet of Things (IoT) is a field that is evolving rapidly, especially in recent years. Therefore there is a possibility of developing even more applications which prove to be useful for many people, whether they have to do with simple functions in automation systems, or with larger scale applications in the industry. For that reason, more and more people want to work in this field, and a large portion of them do not have the technical knowledge to do so. As a result, they end up losing all the potential that IoT can offer.

Model Driven Engineering (MDE), is here to solve this problem, as it can provide these individuals with the development of IoT systems at a more abstract level, which way more user-friendly.

Through this diploma thesis, a user is given the opportunity to model the system they want to implement, using a text tool to describe the devices and their connections. At the same time, it offers automatic source code generation for a variety of IoT devices, tailored to the features the user prefers their system to have, and thus provides a basic implementation of some functions, without having to write any code whatsoever. In fact, the generated code is low-level, as it is designed according to the requirements of a Real Time Operating System (RTOS).