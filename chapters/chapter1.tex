\chapter{Εισαγωγή}
\label{chapter:intro}

\setlength{\parskip}{1em}

Αν και υπήρχε ως ιδέα εδώ και περίπου 50 χρόνια, το Διαδίκτυο των Πραγμάτων (Internet of Things - IoT), και ως όρος αλλά και ως προς τη χρήση του, ήρθε στο επίκεντρο του ενδιαφέροντος τα τελευταία 10 χρόνια, και καθημερινά γίνεται ολοένα και πιο διαδεδομένο. Πλέον μάλιστα ο συνολικός αριθμός συνδεδεμένων συσκευών στο διαδίκτυο είναι μεγαλύτερος από αυτόν του πληθυσμού της Γης.

Μπορεί κανείς να δει την εφαρμογή του σε πολλούς τομείς. Από την δημιουργία ενός "Έξυπνου Σπιτιού" μέχρι και σε κάτι τόσο ουσιώδες όπως την καλύτερη παρακολούθηση ασθενών σε νοσοκομεία και άρα την πιο σωστή περίθαλψή τους. Επίσης χρησιμοπείται και για αυτοματισμούς στη γεωργία, και γενικότερα στη βιομηχανία.

Η ολοένα και μεγαλύτερη διάδοση του IoT, συνεπάγεται και την αξιοποίησή του από κοινό λιγότερο τεχνολογικά καταρτισμένο. Κρίνεται σκόπιμη λοιπόν η ανάπτυξη μεθόδων που θα μετατρέπουν την δημιουργία ενός συστήμος IoT σε διαδικασία πιο φιλική προς τα άτομα αυτά. Εδώ έρχεται να δώσει τη λύση η Μοντελοστρεφής Μηχανική (Model Driven Engineering - MDE), η οποία προσφέρει γρήγορη και πιο αυτοματοποιημένη ανάπτυξη λογισμικού.

\section{Περιγραφή του Προβλήματος}
\label{section:problem_description}

Πέρα από τα προβλήματα που αναλύθηκαν στην προηγούμενη παράγραφο, στα οποία δίνει λύση η MDE, ένα ακόμη σημαντικό θέμα που εμφανίζεται με την ανάπτυξη του κλάδου του IoT είναι η κατασκευή όλο και περισσότερων διαφορετικών IoT συσκευών. Φυσικά, λόγω αυτού από τη μία επεκτείνονται οι δυνατότητες που ένα IoT σύστημα μπορεί να έχει, από την άλλη όμως αυξάνεται η πολυπλοκότητα και ετερογένεια στο IoT.

Υπάρχει πληθώρα IoT συσκευών στην αγορά, όπως π.χ. τα έξυπνα ρολόγια, που διανέμονται έτοιμες για χρήση. Σε αυτές τις περιπτώσεις, οι χρήστες μπορούν να ακολουθήσουν σαφείς οδηγίες χρήσης από τον κατασκευαστή, και άρα πολύ εύκολα να αξιοποιήσουν τις δυνατότητες που η εκάστοτε συσκευή προσφέρει. Επομένως, το πρόβλημα της πολυπλοκότητας δεν εμφανίζεται σε τέτοιου είδους IoT συσκευές.

Στην περίπτωση όμως που κάποιος/α επιθυμεί να αναπτύξει ένα σύστημα με ΙοΤ συσκευές από την αρχή, επειδή π.χ. θέλει να πειραματιστεί ή να υλοποιήσει κάποιες λειτουργίες πιο εξειδικευμένες, τότε απαιτείται μια μεγάλη διαδικασία για την κατασκευή και άρα πολλές γνώσεις. Το πρώτο βήμα είναι η επιλογή των κατάλληλων μικροελεγκτών, αισθητήρων, ενεργοποιητών για την υλοποίηση της ιδέας. Απαιτείται λοιπόν γνώση πάνω στον τρόπο λειτουργίας των συσκευών αυτών, καθώς και στον τρόπο διασύνδεσης και επικοινωνίας τους. Ακολουθεί η ανάπτυξη λογισμικού για την υλοποίηση των επιθυμητών λειτουργιών, κάτι το οποίο από μόνο του σημαίνει πως πρέπει να υπάρχει εμπειρία με προγραμματισμό και πρωτόκολλα επικοινωνίας. Επίσης, σε πολλες περιπτώσεις, στο σύστημα που υλοποιείται απαιτείται η ύπαρξη ιδιοτήτων όπως η ακρίβεια στον χρόνο απόκρισης ή η χαμηλή κατανάλωση ενέργειας. Επομένως, απαιτούνται και οι γνώσεις των ιδιοτήτων των RTOS, καθώς και της κατάλληλης χρήσης τους.

\section{Σκοπός - Συνεισφορά της Διπλωματικής Εργασίας}
\label{section:contribution}

Η παρούσα διπλωματική έχει ως στόχο την ανάπτυξη μιας μηχανής λογισμικού μοντελοστρεφούς λογικής, με την οποία οι χρήστες θα μπορούν να μοντελοποιούν συσκευές καθώς και την σύνδεσή τους. Στη συνέχεια το σύστημα θα υποδεικνύει τη σωστή συνδεσμολογία, και θα παράγονται αυτόματα κάποια προγράμματα που θα υλοποιούν κάποιες βασικές λειτουργίες.
\section{Διάρθρωση της Αναφοράς}
\label{section:layout}

Η διάρθρωση της παρούσας διπλωματικής εργασίας είναι η εξής:

\begin{itemize}
	\item{\textbf{Κεφάλαιο \ref{chapter:sota}:} Γίνεται ανασκόπηση της ερευνητικής δραστηριότητας στον τομέα μέχρι σήμερα.
	}
	\item{\textbf{Κεφάλαιο \ref{chapter:theory}:} Αναλύεται το θεωρητικό υπόβαθρο. 
	}
	\item{\textbf{Κεφάλαιο \ref{chapter:tools}:} Παρουσιάζονται οι διάφορες τεχνικές και τα εργαλεία που χρησιμοποιήθηκαν στις υλοποιήσεις.
	}
	\item{\textbf{Κεφάλαιο \ref{chapter:implementations}:} Παρουσιάζονται τα βήματα της μεθοδολογίας που υλοποιήθηκε.
	}
	\item{\textbf{Κεφάλαιο \ref{chapter:examples}:} Περιγράφονται 3 παραδείγματα εφαρμογής των εργαλείων που αναπτύχθηκαν.
	}
	\item{\textbf{Κεφάλαιο \ref{chapter:conclusions}:} Παρουσιάζονται τα τελικά συμπεράσματα και προτείνονται θέματα για μελλοντική μελέτη, αλλαγές και επεκτάσεις.
	}

\end{itemize}


