\chapter{Εισαγωγή}
\label{chapter:intro}

\setlength{\parskip}{1em}

Αν και υπήρχε ως ιδέα εδώ και περίπου 50 χρόνια, το Διαδίκτυο των Πραγμάτων (Internet of Things - IoT), και ως όρος αλλά και ως προς τη χρήση του, ήρθε στο επίκεντρο του ενδιαφέροντος τα τελευταία 10 χρόνια, και καθημερινά γίνεται ολοένα και πιο διαδεδομένο. Πλέον μάλιστα ο συνολικός αριθμός συνδεδεμένων συσκευών στο διαδίκτυο είναι μεγαλύτερος από αυτόν του πληθυσμού της Γης.

Μπορεί κανείς να δει την εφαρμογή του σε πολλούς τομείς. Από την δημιουργία ενός "Έξυπνου Σπιτιού" μέχρι και σε κάτι τόσο ουσιώδες όπως την καλύτερη παρακολούθηση ασθενών σε νοσοκομεία και άρα την πιο σωστή περίθαλψή τους. Επίσης χρησιμοπείται και για αυτοματισμούς στη γεωργία, και γενικότερα στη βιομηχανία.

Η ολοένα και μεγαλύτερη διάδοση του IoT, συνεπάγεται και την αξιοποίησή του από κοινό λιγότερο τεχνολογικά καταρτισμένο. Κρίνεται σκόπιμη λοιπόν η ανάπτυξη μεθόδων που θα μετατρέπουν την δημιουργία ενός συστήμος IoT σε διαδικασία πιο φιλική προς τα άτομα αυτά. Εδώ έρχεται να δώσει τη λύση η Οδηγούμενη από Μοντέλα Μηχανική (Model Driven Engineering - MDE), η οποία προσφέρει γρήγορη και πιο αυτοματοποιημένη ανάπτυξη λογισμικού.

\section{Περιγραφή του Προβλήματος}
\label{section:problem_description}

Υπάρχει πληθώρα IoT συσκευών στην αγορά, εύκολα διαχειρίσμιμες από όλον τον κόσμο, ανεξαρτήτως των γνώσεών του στον τομέα. Μάλιστα οι εμπορικές αυτές συσκευές προσφέρουν πολλές δυνατότητες και λειτουργίες, και άρα μπορεί ο/η καθένας/καθεμία να τις χρησιμοποιήσει και να καλυφθεί από αυτό.

Στην περίπτωση όμως που κάποιος/α επιθυμεί να αναπτύξει ένα σύστημα με ΙοΤ συσκευές από την αρχή, επειδή π.χ. θέλει να πειραματιστεί ή να υλοποιήσει κάποιες λειτουργίες πιο εξειδικευμένες, τότε απαιτείται μια μεγάλη διαδικασία για την κατασκευή και άρα πολλές γνώσεις. Το πρώτο βήμα είναι η επιλογή των κατάλληλων μικροελεγκτών, αισθητήρων, ενεργοποιητών για την υλοποίηση της ιδέας. Απαιτείται λοιπόν γνώση πάνω στον τρόπο λειτουργίας των συσκευών αυτών, καθώς και στον τρόπο διασύνδεσης και επικοινωνίας τους. Ακολουθεί η ανάπτυξη λογισμικού για την υλοποίηση των επιθυμητών λειτουργιών, κάτι το οποίο από μόνο του σημαίνει πως πρέπει να υπάρχει εμπειρία με χαμηλού επιπέδου προγραμματισμό και πρωτόκολλα επικοινωνίας.

\section{Σκοπός - Συνεισφορά της Διπλωματικής Εργασίας}
\label{section:contribution}

Η παρούσα διπλωματική έχει ως στόχο την ανάπτυξη μιας μηχανής λογισμικού μοντελοστρεφούς λογικής, με την οποία οι χρήστες θα μπορούν να μοντελοποιούν συσκευές καθώς και την σύνδεσή τους. Στη συνέχεια το σύστημα θα υποδεικνύει τη σωστή συνδεσμολογία, και θα παράγονται αυτόματα κάποια προγράμματα που θα υλοποιούν κάποιες βασικές λειτουργίες. Ο παραγόμενος κώδικας θα αφορά συσκευές που τρέχουν το λειτουργικό σύστημα πραγματικού χρόνου RIOT.
\input{./chapters/chapter1_intro/section3_layout.tex}
