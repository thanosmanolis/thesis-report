\section{Υποστηριζόμενες συσκευές}
\label{sec:supported}

Για να υποστηρίζεται κάποια συσκευή από τη διαδικασία, είναι απαραίτητο να έχουν πρώτα συγγραφεί κάποια συγκεκριμένα αρχεία.

Στην περίπτωση κάποιου μικροελεγκτή, χρειάζεται να υπάρχει το αντίστοιχο configuration αρχείο (.hwd) σύμφωνα με το συντακτικό  που αναλύεται στην \autoref{subsec:syntax_device}.

Στην περίπτωση κάποιου περιφερειακού, χρειάζεται να υπάρχει το αντίστοιχο configuration αρχείο (.hwd) σύμφωνα με το συντακτικό  που αναλύεται στην \autoref{subsec:syntax_device}, αλλά και ένα πρότυπο αρχείο κώδικα C, στο οποίο υλοποιούνται κάποιες βασικές λειτουργίες οι οποίες αναλύονται στην \autoref{subsec:c_code}.

Στα πλαίσια της παρούσας διπλωματικής εργασίας, γράφτηκαν πρότυπα αρχεία, και configuration αρχεία (.hwd) για 2 μικροελεγκτές, 3 σένσορες και 1 ενεργοποιητή. Οι συσκευές αυτές είναι οι ακόλουθες:

\begin{itemize}
	\item NODEMCU ESP-32S\footnote{\url{https://docs.ai-thinker.com/_media/esp32/docs/nodemcu-32s_product_specification.pdf}}
	\item WEMOS D1 miniPro\footnote{\url{https://www.wemos.cc/en/latest/d1/d1_mini.html}}
	\item Αισθητήρας αποστασης HC-SR04\footnote{\url{https://www.sparkfun.com/products/15569}}
	\item Αισθητήρας περιβάλλοντος BME680\footnote{\url{https://shop.pimoroni.com/products/bme680-breakout}}
	\item Αισθητήρας περιβάλλοντος MPL3115A2\footnote{\url{https://www.adafruit.com/product/1893}}
	\item Ενεργοποιητής LED NeoPixel Ring\footnote{\url{https://www.adafruit.com/product/1643}}
\end{itemize}