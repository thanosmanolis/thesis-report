\section{Μετασχηματισμός M2M}
\label{sec:transformation}

Αφού δημιουργηθούν τα κατάλληλα αρχεία περιγραφής συσκευών και συνδέσεων, σύμφωνα με τους κανόνες των συντακτικών που αναλύθηκαν στο \autoref{sec:dsl} ξεκινάει η κύρια διαδικασία της παρούσας εργασίας.

Πρώτο βήμα, είναι η δημιουργία μοντέλων για κάθε μια από της συσκευές που χρησιμοποιούνται στην εκάστοτε υλοποίηση, αλλά και για τις μεταξύ τους συνδέσεις. Πριν όμως γίνει η παραγωγή του κατάλληλου κώδικα, που θα αναλυθεί στην επόμενη ενότητα, δημιουργούνται δύο διαγράμματα για το μοντέλο των συνδέσεων, τα οποία βοηθούν στην οπτικοποίηση και άρα καλύτερη αντίληψη από τον χρήστη για τη συνδεσμολογία και ενδοεπικοινωνία του συστήματός του.

Το πρώτο διάγραμμα είναι η παρουσίαση όλων των χαρακτηριστικών των συνδέσεων με τη μορφή οντοτήτων, ενώ το δεύτερο είναι μια οπτικοποίηση των αντιστοιχίσεων των ακροδεκτών με τους οποίους συνδέονται οι συσκευές μεταξύ τους, αλλά και των topic στα οποία επικοινωνεί η εκάστοτε συσκευή.

Για την παραγωγή των διαγραμμάτων αυτών, γίνεται χρήση του εργαλείου PlantUML. Για την παραγωγή των PlantUML αρχείων (βασισμένα στην DSL που το εργαλείο αυτό υποστηρίζει), γίνεται αρχικά ένας M2M μετασχηματισμός του μοντέλου συνδέσεων στο μοντέλο της DSL του PlantUML. Ο μετασχηματισμός αυτός πραγματοποιείται μέσω δύο \textit{python modules}\footnote{\url{https://docs.python.org/3/tutorial/modules.html}} που δημιουργήθηκαν στα πλαίσια της παρούσας διπλωματικής εργασίας. Ένα παράδειγμα από το κάθε διάγραμμά παρουσιάζεται στο \autoref{appendix:diagrams}.