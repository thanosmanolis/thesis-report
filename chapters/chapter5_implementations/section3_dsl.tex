\section{Γραμματική των DSL}
\label{sec:dsl}

Τα μοντέλα που απαρτίζουν την παρούσα εργασία, υπακούουν στους κανόνες που θέτουν τα δύο μετα-μοντέλα που αναλύθηκαν στo \autoref{sec:metamodel_device} και στο \autoref{sec:metamodel_connections}. Για τη δημιουργία τους, αναπτύχθηκαν δύο γλώσσες συγκεκριμένου πεδίου μέσω των οποίων ορίζονται και περιγράφονται οι συσκευές και οι μεταξύ τους συνδέσεις.

Οι γλώσσες αυτές σχεδιάστηκαν με το εργαλείο textX και σε αυτήν την ενότητα περιγράφεται αναλυτικά το συντακτικό τους.

\subsection{Γενικό Συντακτικό}
\label{subsec:syntax}
Και στις δύο γλώσσες που αναπτύχθηκαν, υπάρχουν κάποιοι κοινοί κανόνες συγγραφής.

Αρχικά, κάθε ιδιότητα των μοντέλων δηλώνεται γράφοντας το όνομά της, στη συνέχεια τον χαρακτήρα ":" και τέλος την τιμή της. Για παράδειγμα:

\begin{lstlisting}[title={Παράδειγμα ανάθεσης τιμής σε ιδιότητα}]
	name: value
\end{lstlisting}

Δεύτερος γενικός κανόνας είναι αυτός του τρόπου συγγραφής λίστας. Στην περίπτωση δηλαδή, όπου η τιμή της ιδιότητας είναι μια λίστα από τιμές. Τότε, για κάθε στοιχείο της λίστας γράφουμε αρχικά τον χαρακτήρα "-" και στη συνέχεια τον τύπο του. Για παράδειγμα, τα pins μιας συσκευής μπορεί να είναι power, io\_pin, input\_pin ή output\_pin. Οπότε κάποια από τα pins ενός αισθητήρα θα μπορούσαν να είναι τα εξής.

\begin{lstlisting}[title={Παράδειγμα ορισμού pins αισθητήρα}]
	pins:
		- power:
			name: vcc
			number: 1
			type: 5v
		- input_pin:
			name: data_in
			number: 3
		- output_pin:
			name: data_out
			number: 4
\end{lstlisting}

\subsection{Συντακτικό Συσκευής}
\label{subsec:syntax_device}

Μια συσκευή μπορεί να είναι είτε μικροελεγκτής είτε περιφερειακό (αισθητήρας/ενεργοποιητής). Στην αρχή του αρχείου το πρώτο που δηλώνουμε είναι αυτό, γράφοντας "Board:" για τους μικροελεγκτές και "Peripheral:" για τα περιφερειακά. Τα περισσότερα χαρακτηριστικά είναι κοινά και για τα δύο, επομένως ότι εξηγηθεί παρακάτω εφαρμόζεται με τον ίδιο τρόπο είτε έχουμε μικροελεγκτή είτε περιφερειακό. Στο τέλος θα αναλυθούν ξεχωριστά τα χαρακτηριστικά που δηλώνονται μόνο για μικροελεγκτή και αυτά που δηλώνονται μόνο για περιφερειακό.

Αξίζει να σημειωθεί πως τα χαρακτηριστικά για την κάθε συσκευή, δεν έχει σημασία με ποια σειρά θα δηλωθούν.

\subsubsection{Βασικές Ιδιότητες}
Κάποιες ιδιότητες είναι αρκετά εύκολο να συγγραφούν, καθώς αποτελούν μια απλή ανάθεση τιμής. Ο τρόπος συγγραφής τους είναι ο ακόλουθος:

\begin{lstlisting}
	name: value
	riot_name: value
	vcc: value
	operating_voltage: value
\end{lstlisting}

Το name είναι το όνομα που δίνουμε στη συσκευή και δέχεται ως τιμή αλφαριθμητικό. Το riot\_name είναι το όνομα της συσκευής όπως αναγνωρίζεται από το RIOT, και δέχεται ως τιμή αλφαριθμητικό με επιπλέον επιλογή να περιλαμβάνει και παύλες. Το vcc και το operating\_voltage είναι η τάση τροφοδοσίας και η τάση λειτουργίας της συσκευής αντίστοιχα, και παίρνουν ως τιμή αριθμό κινητής υποδιαστολής.

\subsubsection{Pins}

Για τα pins της εκάστοτε συσκευής, δηλώνουμε το είδος του pin, και ανάλογα το είδος μπορεί να έχει συγκεκριμένα χαρακτηριστικά. Ο γενικός τρόπος σύνταξης είναι ο ακόλουθος (όπου μπορούμε να έχουμε πολλά pins και το καθένα να έχει πολλά χαρακτηριστικά):

\begin{lstlisting}
	pins:
		- pin_type:
			attribute: value
\end{lstlisting}

Το pin\_type μπορεί να είναι ένα εκ των "power", "io\_pin", "input\_pin" και "output\_pin". Σε όλες τις περιπτώσεις, όπου name είναι είναι ένα αλφαριθμητικό που δηλώνει το όνομα του ακροδέκτη, και number ένας ακέραιος αριθμός που υποδηλώνει την θέση του ακροδέκτη στη συσκευή. Επίσης, όπου υπάρχει το vmax υποδηλώνει την μέγιστη τιμή τάσης που μπορει να δεχτεί ο ακροδέκτης (η συγκεκριμένη ιδιότητα δεν είναι υποχρεωτική).

\textbf{\underline{power}}

\begin{lstlisting}
	power:
		name: value
		number: value
		type: value
\end{lstlisting}

Όπου type πρέπει να είναι ένα εκ των "gnd", "5v" και "3v3".

\textbf{\underline{input\_pin}}

\begin{lstlisting}
	input_pin:
		name:value
		number: value
		vmax: value
\end{lstlisting}

\textbf{\underline{output\_pin}}

\begin{lstlisting}
	output_pin:
		name:value
		number: value
		vmax: value
\end{lstlisting}

\textbf{\underline{io\_pin}}

\begin{lstlisting}
	io_pin: -> function1, function2, function3
		name:value
		number: value
		vmax: value
\end{lstlisting}

Το io\_pin πέρα από τα name, number και vmax, έχει μια επιπλέον ιδιότητα, η οποία είναι οι λειτουργίες που μπορεί να έχει. Για να τις ορίσουμε, αρχικά γράφουμε τους χαρακτήρες "->" και στη συνέχεια γράφουμε τις λειτουργίες, χωρίζοντάς τες με το χαρακτήρα "," (εφόσον είναι περισσότερες από μία). Οι λειτουργίες που μπορεί να έχει ένας ακροδέκτης είναι οι ακόλουθες:

\begin{itemize}
	\item gpio
	\item sda: SDA σε I2C διεπαφή
	\item scl: SCL σε I2C διεπαφή
	\item mosi: MOSI σε SPI διεπαφή
	\item miso: MISO σε SPI διεπαφή
	\item sck: SCK σε SPI διεπαφή
	\item cs: CS σε SPI διεπαφή
	\item tx: TX σε UART διεπαφή
	\item rx: RX σε UART διεπαφή
	\item pwm
	\item adc
	\item dac
\end{itemize}

Στην περίπτωση των pin τύπου I2C, SPI, UART και PWM ακολουθεί και ο χαρακτήρας "-" ώστε μετά από αυτόν να μπει ένας αριθμός, ο οποίος συμβολίζει τον αριθμό της διεπαφής στην οποία ανήκει το pin.

Ένα παράδειγμα με μερικά από τα pins ενός μικροελεγκτή θα μπορούσε να είναι το ακόλουθο:

\begin{lstlisting}
	pins:
		...
		...
		- power:
			name: gnd_1
			number: 14
			type: gnd
		- io_pin: -> gpio, mosi-1, adc
			name: p_13
			number: 15
		- io_pin: -> gpio, rx-1
			name: p_9
			number: 16
		- io_pin: -> gpio, tx-1
			name: p_10
			number: 17
		- io_pin: -> gpio
			name: p_11
			number: 18
		- power:
			name: power_5v
			number: 19
			type: 5v
		...
		...
\end{lstlisting}

\subsubsection{Επιπλέον χαρακτηριστικά ενός Board}

\textbf{\underline{Memory}}

\begin{lstlisting}
	memory:
		ram: value unit
		rom: value unit
		flash: value unit
\end{lstlisting}

Όπου value είναι ακέραιος αριθμός που δηλώνει το μέγεθος της μνήμης και unit μπορεί να είναι ένα εκ των "b", "mb", "kb" και "gb". Δεν χρειάζεται να περιλαμβάνονται και τα τρία είδη μνήμης (ram, rom, flash), αλλά τουλάχιστον ένα από αυτά.

\textbf{\underline{Cpu}}

\begin{lstlisting}
	cpu:
		cpu_family: value
		max_freq: value unit
		fpu: value
\end{lstlisting}

Όπου cpu\_family ένα εκ των "ESP32" και "ESP8266". Το max\_freq παίρνει ως τιμή έναν αριθμό, και το unit παίρνει για τιμή ένα εκ των "hz", "ghz" και "mhz". Τέλος το fpu είναι είναι τύπου boolean.
2990
\textbf{\underline{Network}}

\begin{lstlisting}
	network:
		- network_type:
			attribute: value
\end{lstlisting}

Όπου network\_type είναι ένα εκ των "wifi" και "ethernet". Όπως βλέπουμε και παρακάτω, και στις δύο περιπτώσεις υπάρχει το όνομα (name) του δικτύου το οποίο δέχεται ως τιμή ένα αλφαριθμητικό. Στην περίπτωση του wifi έχουμε επιπλεόν την επιλογή freq η οποία δέχεται ως value έναν αριθμό, και ως unit ένα εκ των "hz", "mhz" και "ghz". Το network δεν είναι υποχρεωτικό στοιχείο.

- wifi

\begin{lstlisting}
	wifi:
		name: value
		freq: value unit
\end{lstlisting}

- ethernet

\begin{lstlisting}
	ethernet:
		name: value
\end{lstlisting}

\textbf{\underline{Bluetooth}}

\begin{lstlisting}
	bluetooth:
		version: value
\end{lstlisting}

Η τιμή της έκδοσης (version) είναι ένας αριθμός. Το bluetooth δεν είναι υποχρεωτικό στοιχείο.

\subsubsection{Επιπλέον χαρακτηριστικά ενός Peripheral}

\textbf{\underline{Type}}

\begin{lstlisting}
	type: value
\end{lstlisting}

Το μόνο επιπλέον χαρακτηριστικό που έχει ένα περιφερειακό πέρα από τις βασικές ιδιότητες, είναι ο τύπος, που μπορεί να έχει ως τιμή ένα εκ των "sensor" και "actuator".

\subsection{Συντακτικό Συνδέσεων}
\label{subsec:syntax_connections}

Κάθε μοντέλο βασισμένο στο μετα-μοντέλο συνδέσεων αποτελείται από ένα σύστημα. Το σύστημα αυτό αποτελείται από μία ή περισσότερες συνδέσεις μεταξύ μικροελεγκτών και περιφερειακών. Κάθε μία από τις συσκευές που χρησιμοποιείται στο σύστημα, αρχικά πρέπει να δηλωθεί με την ακόλουθη σύνταξη:

\begin{lstlisting}
	include value
\end{lstlisting}

Όπου value το όνομα της εκάστοτε συσκευής, όπως έχει δοθεί στο configuration file της (αρχείο κατασκευασμένο σύμφωνα με τους κανόνες του συντακτικού που αναλύεται στην \autoref{subsec:syntax_device})).

Στη συνέχεια, για κάθε μία από τις συνδέσεις που υπάρχουν στο σύστημα, γράφεται η λέξη "connection" και ο χαρακτήρας ":" και ακολουθούν οι ιδιότητές της.

\subsubsection{Ιδιότητες}

Αρχικά βλέπουμε τρεις ιδιότητες (name, board, peripheral) οι οποίες έχουν αρκετά απλή σύνταξη, απλής ανάθεσης τιμής.

\begin{lstlisting}
	connection:
		name: value
		board: value (number)
		peripheral: value (number)
\end{lstlisting}

Και οι τρεις ιδιότητες παίρνουν ως value κάποιο αλφαριθμητικό. To board και το peripheral έχουν και την επιλογή (όχι υποχρεωτικό) να έχουν και έναν αναγνωριστικό αριθμό (το number), τοποθετημένο ανάμεσα στους χαρακτήρες "(" και ")". Η δυνατότητα αυτή παρέχεται για την περίπτωση όπου σε πολλαπλές συνδέσεις χρησιμοποιείται συσκευή με το ίδιο όνομα.

\textbf{\underline{power\_connections}}

\begin{lstlisting}
	power_connection:
		- board_pin1 -- peripheral_pin1
		- board_pin2 -- peripheral_pin2
		...
		...
\end{lstlisting}

Η συνδέσεις τροφοδοσίας συντάσσονται ως μια λίστα από αντιστοιχίσεις των pins του μικροελεγκτή, με αυτά του περιφερειακού. Άρα για κάθε σύνδεση pin, αρχικά αναγράφεται το όνομα του pin του μικροελεγκτή, και στη συνέχεια το όνομα του pin του περιφερειακού. Τα δύο ονόματα διαχωρίζονται από τους χαρακτήρες "--". Κάθε στοιχείο της λίστας ξεκινάει με τον χαρακτήρα "-".

\textbf{\underline{hw\_connections}}

Η συνδέσεις διεπαφών υλικού, έχουν διαφορετικό τρόπο σύνταξης ανάλογα με τον τύπο της σύνδεσης. Οι τύποι σύνδεσης που υποστηρίζονται είναι οι ακόλουθοι:

\begin{itemize}
	\item gpio
	\item i2c
	\item spi
	\item uart
\end{itemize}	

Ωστόσο, σε όλες τις περιπτώσεις η σύνταξη ξεκινάει με τον χαρακτήρα "-", τον τύπο σύνδεσης και τέλος τον χαρακτήρα ":", όπως φαίνεται παρακάτω (όπου type ο τύπος):

\begin{lstlisting}
	- type:
\end{lstlisting}

$\rightarrow$ \underline{\textit{gpio}}

\begin{lstlisting}
	hw_connections:
		...
		...
		- gpio: board_pin1 -- peripheral_pin1
		- gpio: board_pin2 -- peripheral_pin2
		...
		...
\end{lstlisting}

Οι συνδέσεις gpio συντάσσονται ακριβώς όπως οι συνδέσεις τροφοδοσίας, με τη μόνη προσθήκη του "gpio:" το οποίο είναι ο τύπος σύνδεσης όπως αναφέρθηκε προηγουμένως.

$\rightarrow$ \underline{\textit{i2c}}

\begin{lstlisting}
	hw_connections:
		...
		...
		- i2c: 
			sda: board_pin -- peripheral_pin
			scl: board_pin -- peripheral_pin
			slave_address: value
		...
		...
\end{lstlisting}

Οι συνδέσεις i2c συντάσσονται αρχικά με τον τύπο της σύνδεσης ("- i2c:) και στη συνέχεια μια λίστα με τα στοιχεία της σύνδεσεις αυτής. Τα sda και scl συντάσσονται όπως και το gpio, ενώ το slave\_address παίρνει τιμή έναν δεκαεξαδικό αριθμό της μορφής "0x00" (πρώτα οι χαρακτήρες "0x" και στη συνέχεια ο δεκαεξαδικός).

$\rightarrow$ \underline{\textit{spi}}

\begin{lstlisting}
	hw_connections:
		...
		...
		- spi: 
			mosi: board_pin -- peripheral_pin
			miso: board_pin -- peripheral_pin
			sck: board_pin -- peripheral_pin
			cs: board_pin -- peripheral_pin
		...
		...
\end{lstlisting}

Οι συνδέσεις spi συντάσσονται αρχικά με τον τύπο της σύδεσης ("- spi:") και στη συνέχεια όπως και οι συνδέσεις i2c, με τη διαφορά ότι δεν υπάρχει slave\_address, και τα είδη των pin αντί για sda και scl είναι mosi, miso, sck και cs.

$\rightarrow$ \underline{\textit{uart}}

\begin{lstlisting}
	hw_connections:
		...
		...
		- uart: 
			tx: board_pin -- peripheral_pin
			rx: board_pin -- peripheral_pin
			baudrate: value
		...
		...
\end{lstlisting}

Οι συνδέσεις uart συντάσσονται αρχικά με τον τύπο της σύδεσης ("- uart:") και στη συνέχεια όπως και οι συνδέσεις i2c, με τη διαφορά ότι αντί για slave\_address υπάρχει το στοιχείο baudrate, που παίρνει ως τιμή κάποιον ακέραιο αριθμό, και τα είδη των pin αντί για sda και scl είναι tx και rx.

\textbf{\underline{communication\_endpoint}}

\begin{lstlisting}
	communication_endpoint:
		topic: value
		wifi_ssid: value
		wifi_passwd: value
		address: value
		port: value
		msg: value
		frequency: value unit
\end{lstlisting}

Τελευταία ιδιότητας μιας σύνδεσης είναι τα χαρακτηριστικά για τη σύνδεση σε κάποιον broker. Τα topic, wifi\_ssid και wifi\_password παίρνουν ως τιμή κάποια αλληλουχία γραμμάτων, αριθμών και συμβόλων χωρίς κενά. Το στοιχείο address παίρνει ως τιμή μια διεύθυνση τύπου IPv6 καιτο port κάποιον ακέραιο αριθμό. Το frequency (το οποίο δεν είναι υποχρεωτικό) παίρνει ως τιμή κάποιον ακέραιο και ως unit ένα εκ των "hz", "khz", "mhz" και "ghz".

Τέλος το msg είναι το είδος του μηνύματος που θα γίνει publish ή θα ληφθεί από κάποιον subscriber και μπορεί να είναι ένα εκ των "Distance", "Temperature", "Humidity", "Gas", "Pressure", "Env", "Acceleration", "Motor\_Controller", "Leds\_Controller", "Servo\_Controller".