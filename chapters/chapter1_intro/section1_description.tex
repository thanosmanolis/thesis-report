\section{Περιγραφή του Προβλήματος}
\label{section:problem_description}

Υπάρχει πληθώρα IoT συσκευών στην αγορά, εύκολα διαχειρίσμιμες από όλον τον κόσμο, ανεξαρτήτως των γνώσεών του στον τομέα. Μάλιστα οι εμπορικές αυτές συσκευές προσφέρουν πολλές δυνατότητες και λειτουργίες, και άρα μπορεί ο/η καθένας/καθεμία να τις χρησιμοποιήσει και να καλυφθεί από αυτό.

Στην περίπτωση όμως που κάποιος/α επιθυμεί να αναπτύξει ένα σύστημα με ΙοΤ συσκευές από την αρχή, επειδή π.χ. θέλει να πειραματιστεί ή να υλοποιήσει κάποιες λειτουργίες πιο εξειδικευμένες, τότε απαιτείται μια μεγάλη διαδικασία για την κατασκευή και άρα πολλές γνώσεις. Το πρώτο βήμα είναι η επιλογή των κατάλληλων μικροελεγκτών, αισθητήρων, ενεργοποιητών για την υλοποίηση της ιδέας. Απαιτείται λοιπόν γνώση πάνω στον τρόπο λειτουργίας των συσκευών αυτών, καθώς και στον τρόπο διασύνδεσης και επικοινωνίας τους. Ακολουθεί η ανάπτυξη λογισμικού για την υλοποίηση των επιθυμητών λειτουργιών, κάτι το οποίο από μόνο του σημαίνει πως πρέπει να υπάρχει εμπειρία με χαμηλού επιπέδου προγραμματισμό και πρωτόκολλα επικοινωνίας.
